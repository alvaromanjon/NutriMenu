\apendice{Especificación de diseño}

\section{Introducción}

\section{Diseño de datos}

\section{Diseño procedimental}

Uno de los grandes cambios que se han realizado en este proyecto respecto a la iteración anterior es el desarrollo de una API REST completa para trabajar e interactuar con los datos. A continuación voy a proceder a detallar y documentar cada uno de los distintos endpoints a los cuáles se puede hacer llamadas HTTP:

% TODO Indicar cómo hacer llamadas a la API %
% TODO Indicar el endpoint al que hacer las llamadas %

\subsection{Empresas}

\begin{apiRoute}{get}{/empresas}{Devuelve todas las empresas existentes}
	
	\begin{routeParameter}
		\routeParamItem{id\_empresa}{Devuelve la empresa con el ID correspondiente}
		\routeParamItem{nombre}{Devuelve todas las empresas cuyo nombre contengan el valor pasado por parámetro}
		\routeParamItem{email}{Devuelve la empresa con el email correspondiente}
		\routeParamItem{telefono}{Devuelve la empresa con el teléfono correspondiente}
		\routeParamItem{cif}{Devuelve la empresa con el CIF correspondiente}
	\end{routeParameter}
	
	\begin{routeResponse}{application/json}
		\begin{routeResponseItem}{200}{ok}
			\begin{routeResponseItemBody}
{
	"id": 1,
	"nombre": "UBU",
	"email": "cafeterias@ubu.es",
	"direccion": "Calle Don Juan de Austria 1, 09001 Burgos",
	"telefono": "947258060",
	"cif": "123456789"
}
			\end{routeResponseItemBody}
		\end{routeResponseItem}
		\begin{routeResponseItem}{400}{EntityDoesntExistsException: No existe una empresa con id}
			\begin{routeResponseItemBody}
{
    "timestamp": "2023-09-19T17:18:27.228+00:00",
    "status": 400,
    "error": "Bad Request",
    "trace": "ooo.alvar.nutrimenu.apirest.excepciones...",
    "message": "No existe una empresa con id 3",
    "path": "/empresas"
}
			\end{routeResponseItemBody}
		\end{routeResponseItem}
	\end{routeResponse}
	
\end{apiRoute}

\begin{apiRoute}{post}{/empresas}{Añade una nueva empresa}
	\begin{routeParameter}
		\noRouteParameter{No hay parámetros}
	\end{routeParameter}
	\begin{routeRequest}{application/json}
		\begin{routeRequestBody}
{
	"nombre": "Vips",
	"email": "empresa@vips.com",
	"direccion": "Puerta del Sol, Madrid",
	"telefono": "900123123",
	"cif": "123456787"
}
		\end{routeRequestBody}
	\end{routeRequest}
	\begin{routeResponse}{application/json}
		\begin{routeResponseItem}{200}{ok}
			\begin{routeResponseItemBody}
{
	"id": 3,
	"nombre": "Vips",
	"email": "empresa@vips.com",
	"direccion": "Puerta del Sol, Madrid",
	"telefono": "900123123",
	"cif": "123456787"
}
			\end{routeResponseItemBody}
		\end{routeResponseItem}
	\end{routeResponse}
\end{apiRoute}

\begin{apiRoute}{put}{/empresas}{Actualiza la información de una empresa ya existente}
	\begin{routeParameter}
		\routeParamItem{id\_empresa}{Modifica cualquier valor de la empresa con el ID correspondiente}
	\end{routeParameter}
	
	\begin{routeRequest}{application/json}
		\begin{routeRequestBody}
{
	"nombre": "Universidad de Burgos"
}
		\end{routeRequestBody}
	\end{routeRequest}
	\begin{routeResponse}{application/json}
		\begin{routeResponseItem}{200}{ok}
			\begin{routeResponseItemBody}
{
	"id": 1,
	"nombre": "UBU",
	"email": "cafeterias@ubu.es",
	"direccion": "Calle Don Juan de Austria 1, 09001 Burgos",
	"telefono": "947258060",
	"cif": "123456789"
}
			\end{routeResponseItemBody}
		\end{routeResponseItem}
		\begin{routeResponseItem}{400}{EntityDoesntExistsException: No existe una empresa con id}
			\begin{routeResponseItemBody}
{
    "timestamp": "2023-09-19T17:18:27.228+00:00",
    "status": 400,
    "error": "Bad Request",
    "trace": "ooo.alvar.nutrimenu.apirest.excepciones...",
    "message": "No existe una empresa con id 3",
    "path": "/empresas"
}
			\end{routeResponseItemBody}
		\end{routeResponseItem}
	\end{routeResponse}
\end{apiRoute}

\begin{apiRoute}{delete}{/empresas}{Elimina una empresa de la base de datos}
	\begin{routeParameter}
		\routeParamItem{id\_empresa}{Elimina la empresa con el ID correspondiente}
	\end{routeParameter}
	\begin{routeResponse}{application/json}
		\begin{routeResponseItem}{200}{ok}
			\begin{routeResponseItemBody}
Empresa con id eliminada correctamente	
			\end{routeResponseItemBody}
		\end{routeResponseItem}
		
\begin{routeResponseItem}{400}{EntityDoesntExistsException: No existe una empresa con id}
			\begin{routeResponseItemBody}
{
    "timestamp": "2023-09-19T17:18:27.228+00:00",
    "status": 400,
    "error": "Bad Request",
    "trace": "ooo.alvar.nutrimenu.apirest.excepciones...",
    "message": "No existe una empresa con id 3",
    "path": "/empresas"
}
			\end{routeResponseItemBody}
		\end{routeResponseItem}
		
	\end{routeResponse}
\end{apiRoute}


\section{Diseño arquitectónico}


