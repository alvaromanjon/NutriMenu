\apendice{Especificación de diseño}

\section{Introducción}

\section{Diseño de datos}

\section{Diseño procedimental}

Uno de los grandes cambios que se han realizado en este proyecto respecto a la iteración anterior es el desarrollo de una API REST completa para trabajar e interactuar con los datos. A continuación voy a proceder a detallar y documentar cada uno de los distintos endpoints a los cuáles se puede hacer llamadas HTTP:

% TODO Indicar cómo hacer llamadas a la API %
% TODO Indicar el endpoint al que hacer las llamadas %

\subsection{Empresas}

\begin{apiRoute}{get}{/empresas}{Devuelve todas las empresas existentes}
	
	\begin{routeParameter}
		\routeParamItem{id\_empresa}{Devuelve la empresa con el ID correspondiente}
		\routeParamItem{nombre}{Devuelve todas las empresas cuyo nombre contengan el valor pasado por parámetro}
		\routeParamItem{email}{Devuelve la empresa con el email correspondiente}
		\routeParamItem{telefono}{Devuelve la empresa con el teléfono correspondiente}
		\routeParamItem{cif}{Devuelve la empresa con el CIF correspondiente}
	\end{routeParameter}
	
	\begin{routeResponse}{application/json}
		\begin{routeResponseItem}{200}{ok}
			\begin{routeResponseItemBody}
{
	"id": 1,
	"nombre": "UBU",
	"email": "cafeterias@ubu.es",
	"direccion": "Calle Don Juan de Austria 1, 09001 Burgos",
	"telefono": "947258060",
	"cif": "123456789"
}
			\end{routeResponseItemBody}
		\end{routeResponseItem}
		\begin{routeResponseItem}{400}{Bad Request: EntityDoesntExistsException}
			\begin{routeResponseItemBody}
{
    "timestamp": "2023-09-19T17:18:27.228+00:00",
    "status": 400,
    "error": "Bad Request",
    "trace": "ooo.alvar.nutrimenu.apirest.excepciones...",
    "message": "No existe una empresa con id 3",
    "path": "/empresas"
}
			\end{routeResponseItemBody}
		\end{routeResponseItem}
	\end{routeResponse}
	
\end{apiRoute}

\begin{apiRoute}{post}{/empresas}{Añade una nueva empresa}
	\begin{routeParameter}
		\noRouteParameter{No hay parámetros}
	\end{routeParameter}
	\begin{routeRequest}{application/json}
		\begin{routeRequestBody}
{
	"nombre": "Vips",
	"email": "empresa@vips.com",
	"direccion": "Puerta del Sol, Madrid",
	"telefono": "900123123",
	"cif": "123456787"
}
		\end{routeRequestBody}
	\end{routeRequest}
	\begin{routeResponse}{application/json}
		\begin{routeResponseItem}{200}{ok}
			\begin{routeResponseItemBody}
{
	"id": 3,
	"nombre": "Vips",
	"email": "empresa@vips.com",
	"direccion": "Puerta del Sol, Madrid",
	"telefono": "900123123",
	"cif": "123456787"
}
			\end{routeResponseItemBody}
		\end{routeResponseItem}
	\end{routeResponse}
\end{apiRoute}

\begin{apiRoute}{put}{/empresas}{Actualiza la información de una empresa ya existente}
	\begin{routeParameter}
		\routeParamItem{id\_empresa}{Modifica cualquier valor de la empresa con el ID correspondiente}
	\end{routeParameter}
	
	\begin{routeRequest}{application/json}
		\begin{routeRequestBody}
{
	"nombre": "Universidad de Burgos"
}
		\end{routeRequestBody}
	\end{routeRequest}
	\begin{routeResponse}{application/json}
		\begin{routeResponseItem}{200}{ok}
			\begin{routeResponseItemBody}
{
	"id": 1,
	"nombre": "UBU",
	"email": "cafeterias@ubu.es",
	"direccion": "Calle Don Juan de Austria 1, 09001 Burgos",
	"telefono": "947258060",
	"cif": "123456789"
}
			\end{routeResponseItemBody}
		\end{routeResponseItem}
		\begin{routeResponseItem}{400}{Bad Request: EntityDoesntExistsException}
			\begin{routeResponseItemBody}
{
    "timestamp": "2023-09-19T17:18:27.228+00:00",
    "status": 400,
    "error": "Bad Request",
    "trace": "ooo.alvar.nutrimenu.apirest.excepciones...",
    "message": "No existe una empresa con id 3",
    "path": "/empresas"
}
			\end{routeResponseItemBody}
		\end{routeResponseItem}
	\end{routeResponse}
\end{apiRoute}

\begin{apiRoute}{delete}{/empresas}{Elimina una empresa de la base de datos}
	\begin{routeParameter}
		\routeParamItem{id\_empresa}{Elimina la empresa con el ID correspondiente}
	\end{routeParameter}
	\begin{routeResponse}{application/json}
		\begin{routeResponseItem}{200}{ok}
			\begin{routeResponseItemBody}
Empresa con id eliminada correctamente	
			\end{routeResponseItemBody}
		\end{routeResponseItem}
		
\begin{routeResponseItem}{400}{Bad Request: EntityDoesntExistsException}
			\begin{routeResponseItemBody}
{
    "timestamp": "2023-09-19T17:18:27.228+00:00",
    "status": 400,
    "error": "Bad Request",
    "trace": "ooo.alvar.nutrimenu.apirest.excepciones...",
    "message": "No existe una empresa con id 3",
    "path": "/empresas"
}
			\end{routeResponseItemBody}
		\end{routeResponseItem}
		
	\end{routeResponse}
\end{apiRoute}

\subsection{Locales}

\begin{apiRoute}{get}{/locales}{Devuelve todos los locales existentes}
	
	\begin{routeParameter}
		\routeParamItem{id\_local}{Devuelve el local con el ID correspondiente}
		\routeParamItem{id\_empresa}{Devuelve todos los locales correspondientes a la empresa con el ID correspondiente}
		\routeParamItem{nombre}{Devuelve todos los locales cuyo nombre contengan el valor pasado por parámetro}
		\routeParamItem{email}{Devuelve el local con el email correspondiente}
		\routeParamItem{telefono}{Devuelve el local con el teléfono correspondiente}
	\end{routeParameter}
	
	\begin{routeResponse}{application/json}
		\begin{routeResponseItem}{200}{ok}
			\begin{routeResponseItemBody}
{
	"id": 1,
	"nombre": "Escuela Politecnica",
	"email": "eps@ubu.es",
	"direccion": "Avenida Cantabria",
	"telefono": "947123123",
	"empresa": {
		"id": 1,
		"nombre": "Universidad de Burgos",
		"email": "cafeterias@ubu.es",
		"direccion": "Calle Don Juan de Austria 1, 09001 Burgos",
		"telefono": "9002375432",
		"cif": "1113111"
        }
    }
			\end{routeResponseItemBody}
		\end{routeResponseItem}
		\begin{routeResponseItem}{400}{Bad Request: EntityDoesntExistsException}
			\begin{routeResponseItemBody}
{
    "timestamp": "2023-09-19T17:18:27.228+00:00",
    "status": 400,
    "error": "Bad Request",
    "trace": "ooo.alvar.nutrimenu.apirest.excepciones...",
    "message": "No existe una empresa con id 3",
    "path": "/locales"
}
			\end{routeResponseItemBody}
		\end{routeResponseItem}
	\end{routeResponse}
	
\end{apiRoute}

\begin{apiRoute}{post}{/locales}{Añade un nuevo local}
	\begin{routeParameter}
		\routeParamItem{id\_empresa}{Indica a qué empresa pertenece el local}
	\end{routeParameter}
	\begin{routeRequest}{application/json}
		\begin{routeRequestBody}
{
	"nombre": "Cafeteria EPS",
	"email": "eps@ubu.es",
	"direccion": "Avenida Cantabria",
	"telefono": "947123954"
}
		\end{routeRequestBody}
	\end{routeRequest}
	\begin{routeResponse}{application/json}
		\begin{routeResponseItem}{200}{ok}
			\begin{routeResponseItemBody}
{
    "id": 3,
    "nombre": "Cafeteria EPS",
    "email": "eps@ubu.es",
    "direccion": "Avenida Cantabria",
    "telefono": "947123954",
    "empresa": {
        "id": 1,
        "nombre": "Universidad de Burgos",
        "email": "cafeterias@ubu.es",
        "direccion": "Calle Don Juan de Austria 1, 09001 Burgos",
        "telefono": "9002375432",
        "cif": "1113111"
    }
}
			\end{routeResponseItemBody}
		\end{routeResponseItem}
		\begin{routeResponseItem}{400}{Bad Request: EntityDoesntExistsException}
			\begin{routeResponseItemBody}
{
    "timestamp": "2023-09-19T17:18:27.228+00:00",
    "status": 400,
    "error": "Bad Request",
    "trace": "ooo.alvar.nutrimenu.apirest.excepciones...",
    "message": "No existe una empresa con id 3",
    "path": "/locales"
}
			\end{routeResponseItemBody}
		\end{routeResponseItem}
	\end{routeResponse}
\end{apiRoute}

\begin{apiRoute}{put}{/locales}{Actualiza la información de un local ya existente}
	\begin{routeParameter}
		\routeParamItem{id\_local}{Modifica cualquier valor del local con el ID correspondiente}
	\end{routeParameter}
	
	\begin{routeRequest}{application/json}
		\begin{routeRequestBody}
{
	"nombre": "Cafeteria Economicas"
}
		\end{routeRequestBody}
	\end{routeRequest}
	\begin{routeResponse}{application/json}
		\begin{routeResponseItem}{200}{ok}
			\begin{routeResponseItemBody}
{
    "id": 3,
    "nombre": "Cafeteria Economicas",
    "email": "eps@ubu.es",
    "direccion": "Avenida Cantabria",
    "telefono": "947123954",
    "empresa": {
        "id": 1,
        "nombre": "Universidad de Burgos",
        "email": "cafeterias@ubu.es",
        "direccion": "Calle Don Juan de Austria 1, 09001 Burgos",
        "telefono": "9002375432",
        "cif": "1113111"
    }
}
			\end{routeResponseItemBody}
		\end{routeResponseItem}
		\begin{routeResponseItem}{400}{Bad Request: EntityDoesntExistsException}
			\begin{routeResponseItemBody}
{
    "timestamp": "2023-09-19T17:18:27.228+00:00",
    "status": 400,
    "error": "Bad Request",
    "trace": "ooo.alvar.nutrimenu.apirest.excepciones...",
    "message": "No existe un local con id 3",
    "path": "/locales"
}
			\end{routeResponseItemBody}
		\end{routeResponseItem}
	\end{routeResponse}
\end{apiRoute}

\begin{apiRoute}{put}{/add/local/menus}{Añade un menú a un local}
	\begin{routeParameter}
		\routeParamItem{id\_local}{ID del local al que se le va a añadir un menú}
		\routeParamItem{id\_menu}{ID del menú que va a ser añadido al local}
	\end{routeParameter}
	
	\begin{routeResponse}{application/json}
		\begin{routeResponseItem}{200}{ok}
			\begin{routeResponseItemBody}
{
    "id": 1,
    "nombre": "Menu del miercoles",
    "descripcion": "Menu del martes",
    "fechaCreacion": "2023-09-19T19:00:14.998139Z",
    "fechaModificacion": "2023-09-19T19:22:57.314517Z",
    "platos": [
        {
            "id": 1,
            "tipoPlato": "PRIMER_PLATO",
            "nombre": "Kevin Bacon",
            "descripcion": "Hamburguesa rellena de bacon",
            "fechaCreacion": "2023-09-19T19:00:59.387553Z",
            "fechaModificacion": "2023-09-19T19:00:59.387556Z"
        }
    ]
}
			\end{routeResponseItemBody}
		\end{routeResponseItem}
		\begin{routeResponseItem}{400}{Bad Request: EntityDoesntExistsException}
			\begin{routeResponseItemBody}
{
    "timestamp": "2023-09-19T17:18:27.228+00:00",
    "status": 400,
    "error": "Bad Request",
    "trace": "ooo.alvar.nutrimenu.apirest.excepciones...",
    "message": "No existe un local con id 3",
    "path": "/add/local/menus"
}
			\end{routeResponseItemBody}
		\end{routeResponseItem}
	\end{routeResponse}
\end{apiRoute}

\begin{apiRoute}{delete}{/locales}{Elimina un local de la base de datos}
	\begin{routeParameter}
		\routeParamItem{id\_local}{Elimina el local con el ID correspondiente}
	\end{routeParameter}
	\begin{routeResponse}{application/json}
		\begin{routeResponseItem}{200}{ok}
			\begin{routeResponseItemBody}
Local con id eliminado correctamente	
			\end{routeResponseItemBody}
		\end{routeResponseItem}
		
\begin{routeResponseItem}{400}{Bad Request: EntityDoesntExistsException}
			\begin{routeResponseItemBody}
{
    "timestamp": "2023-09-19T17:18:27.228+00:00",
    "status": 400,
    "error": "Bad Request",
    "trace": "ooo.alvar.nutrimenu.apirest.excepciones...",
    "message": "No existe un local con id 3",
    "path": "/locales"
}
			\end{routeResponseItemBody}
		\end{routeResponseItem}
		
	\end{routeResponse}
\end{apiRoute}

\subsection{Menús}

\begin{apiRoute}{get}{/menus}{Devuelve todos los menús existentes}
	
	\begin{routeParameter}
		\routeParamItem{id\_menu}{Devuelve el menú con el ID correspondiente}
		\routeParamItem{id\_local}{Devuelve todos los menús del local con el ID correspondiente}
	\end{routeParameter}
	
	\begin{routeResponse}{application/json}
		\begin{routeResponseItem}{200}{ok}
			\begin{routeResponseItemBody}
{
	"id": 1,
	"nombre": "Menu del martes",
	"descripcion": "Menu del martes",
	"fechaCreacion": "2023-09-19T19:00:14.998139Z",
	"fechaModificacion": "2023-09-19T19:00:14.998140Z",
	"platos": []
}
			\end{routeResponseItemBody}
		\end{routeResponseItem}
		\begin{routeResponseItem}{400}{Bad Request: EntityDoesntExistsException}
			\begin{routeResponseItemBody}
{
    "timestamp": "2023-09-19T17:18:27.228+00:00",
    "status": 400,
    "error": "Bad Request",
    "trace": "ooo.alvar.nutrimenu.apirest.excepciones...",
    "message": "No existe un menu con id 3",
    "path": "/menus"
}
			\end{routeResponseItemBody}
		\end{routeResponseItem}
		\begin{routeResponseItem}{400}{Bad Request: LackOfParametersException}
			\begin{routeResponseItemBody}
No se ha especificado ningun parametro de busqueda
			\end{routeResponseItemBody}
		\end{routeResponseItem}
	\end{routeResponse}
	
\end{apiRoute}

\begin{apiRoute}{post}{/menus}{Añade un nuevo menú}
	\begin{routeParameter}
		\routeParamItem{id\_local}{Indica a qué local pertenece el menú}
	\end{routeParameter}
	\begin{routeRequest}{application/json}
		\begin{routeRequestBody}
{
	"nombre": "Menu del martes",
	"descripcion": "Menu del martes"
}
		\end{routeRequestBody}
	\end{routeRequest}
	\begin{routeResponse}{application/json}
		\begin{routeResponseItem}{200}{ok}
			\begin{routeResponseItemBody}
{
	"id": 1,
	"nombre": "Menu del martes",
	"descripcion": "Menu del martes",
	"fechaCreacion": "2023-09-19T19:00:14.998139Z",
	"fechaModificacion": "2023-09-19T19:00:14.998140Z",
	"platos": []
}
			\end{routeResponseItemBody}
		\end{routeResponseItem}
		\begin{routeResponseItem}{400}{Bad Request: EntityDoesntExistsException}
			\begin{routeResponseItemBody}
{
    "timestamp": "2023-09-19T17:18:27.228+00:00",
    "status": 400,
    "error": "Bad Request",
    "trace": "ooo.alvar.nutrimenu.apirest.excepciones...",
    "message": "No existe un local con id 3",
    "path": "/menus"
}
			\end{routeResponseItemBody}
		\end{routeResponseItem}
	\end{routeResponse}
\end{apiRoute}

\begin{apiRoute}{put}{/menus}{Actualiza la información de un menú ya existente}
	\begin{routeParameter}
		\routeParamItem{id\_empresa}{Modifica cualquier valor del menú con el ID correspondiente}
	\end{routeParameter}
	
	\begin{routeRequest}{application/json}
		\begin{routeRequestBody}
{
	"nombre": "Menu del miercoles"
}
		\end{routeRequestBody}
	\end{routeRequest}
	\begin{routeResponse}{application/json}
		\begin{routeResponseItem}{200}{ok}
			\begin{routeResponseItemBody}
{
	"id": 1,
	"nombre": "Menu del miercoles",
	"descripcion": "Menu del martes",
	"fechaCreacion": "2023-09-19T19:00:14.998139Z",
	"fechaModificacion": "2023-09-19T19:00:14.998140Z",
	"platos": []
}
			\end{routeResponseItemBody}
		\end{routeResponseItem}
		\begin{routeResponseItem}{400}{Bad Request: EntityDoesntExistsException}
			\begin{routeResponseItemBody}
{
    "timestamp": "2023-09-19T17:18:27.228+00:00",
    "status": 400,
    "error": "Bad Request",
    "trace": "ooo.alvar.nutrimenu.apirest.excepciones...",
    "message": "No existe un menu con id 3",
    "path": "/menus"
}
			\end{routeResponseItemBody}
		\end{routeResponseItem}
	\end{routeResponse}
\end{apiRoute}

\begin{apiRoute}{put}{/add/menu/platos}{Añade un plato a un menú}
	\begin{routeParameter}
		\routeParamItem{id\_menu}{ID del menú al que se le va a añadir un plato}
		\routeParamItem{id\_plato}{ID del plato que va a ser añadido al menú}
	\end{routeParameter}
	
	\begin{routeResponse}{application/json}
		\begin{routeResponseItem}{200}{ok}
			\begin{routeResponseItemBody}
{
    "id": 1,
    "nombre": "Menu del miercoles",
    "descripcion": "Menu del martes",
    "fechaCreacion": "2023-09-19T19:00:14.998139Z",
    "fechaModificacion": "2023-09-19T19:22:57.314517Z",
    "platos": [
        {
            "id": 1,
            "tipoPlato": "PRIMER_PLATO",
            "nombre": "Kevin Bacon",
            "descripcion": "Hamburguesa rellena de bacon",
            "fechaCreacion": "2023-09-19T19:00:59.387553Z",
            "fechaModificacion": "2023-09-19T19:00:59.387556Z"
        }
    ]
}
			\end{routeResponseItemBody}
		\end{routeResponseItem}
		\begin{routeResponseItem}{400}{Bad Request: EntityDoesntExistsException}
			\begin{routeResponseItemBody}
{
    "timestamp": "2023-09-19T17:18:27.228+00:00",
    "status": 400,
    "error": "Bad Request",
    "trace": "ooo.alvar.nutrimenu.apirest.excepciones...",
    "message": "No existe un menu con id 3",
    "path": "/add/menu/platos"
}
			\end{routeResponseItemBody}
		\end{routeResponseItem}
	\end{routeResponse}
\end{apiRoute}

\begin{apiRoute}{delete}{/menus}{Elimina un menú de la base de datos}
	\begin{routeParameter}
		\routeParamItem{id\_menu}{Elimina el menú con el ID correspondiente}
	\end{routeParameter}
	\begin{routeResponse}{application/json}
		\begin{routeResponseItem}{200}{ok}
			\begin{routeResponseItemBody}
Menu con id eliminado correctamente	
			\end{routeResponseItemBody}
		\end{routeResponseItem}
		
\begin{routeResponseItem}{400}{Bad Request: EntityDoesntExistsException}
			\begin{routeResponseItemBody}
{
    "timestamp": "2023-09-19T17:18:27.228+00:00",
    "status": 400,
    "error": "Bad Request",
    "trace": "ooo.alvar.nutrimenu.apirest.excepciones...",
    "message": "No existe un menu con id 3",
    "path": "/menus"
}
			\end{routeResponseItemBody}
		\end{routeResponseItem}
		
	\end{routeResponse}
\end{apiRoute}

\section{Diseño arquitectónico}


