\apendice{Documentación técnica de programación}

\section{Introducción}

\section{Estructura de directorios}

\section{Manual del programador}

\section{Compilación, despliegue y ejecución del proyecto}

Para realizar la compilación, despliegue y ejecución del proyecto es imprescindible disponer de \textbf{Docker CLI} y \textbf{Docker Compose} instalados en el equipo. La mejor forma de instalarlos es mediante \textbf{Docker Desktop}, ya que se encarga de instalar en el equipo el \textit{daemon} de Docker, Docker CLI, Docker Compose, y demás dependencias y herramientas que no vamos a utilizar para este proyecto, pero que nos pueden ayudar en el futuro. Docker Desktop es compatible con Windows, macOS y Linux, y soporta tanto arquitecturas x86 como ARM64 (Apple Silicon).

\subsection{Instalación de Docker Desktop}

\imagen{Docker/docker-download}{Página de descarga de Docker Desktop}

Para realizar la instalación de Docker Desktop tan sólo debemos dirigirnos a \url{https://docs.docker.com/get-docker/}, seleccionar la plataforma deseada, y descargar el instalador.

\imagen{Docker/docker-installation-mac}{Instalación de Docker Desktop en Mac}

Una vez ejecutemos el instalador (en caso de macOS simplemente se debe arrastrar la aplicación a la carpeta de Aplicaciones), y hayamos seguido todos los pasos hasta finalizar la instalación, nos encontraremos con el panel principal de Docker Desktop. 

\imagen{Docker/docker-main-window}{Ventana principal de Docker Desktop}

En esta ventana podemos ver todos los contenedores que están ejecutándose actualmente en el sistema, así como las imágenes descargadas y los volúmenes creados.
 
\section{Pruebas del sistema}
