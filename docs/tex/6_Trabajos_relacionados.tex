\capitulo{6}{Trabajos relacionados}

En este apartado se van a explorar algunos proyectos y aplicaciones que comparten objetivos similares al de NutriMenu. Este análisis permite entender el contexto en el que se se encuentra nuestro proyecto comparado con el resto de soluciones existentes en el mercado.

\section{MyFitnessPal}

MyFitnessPal es una aplicación web y móvil enfocada principalmente en el seguimiento de la dieta y el ejercicio, permitiendo establecer \textbf{distintas metas en función de los objetivos} (perder peso, mantenerlo, o ganar masa muscular).

Esta aplicación dispone de \textbf{su propia base de datos nutricional}, cubriendo desde alimentos hasta productos comerciales u opciones de restaurantes. 

También dispone de una \textbf{función para crear comidas recordadas}, que funcionaría de manera similar al como actúa NutriMenu a la hora de crear platos, aunque en este caso esto está más enfocado a los propios consumidores, en vez de a los propios locales.

Una de las ventajas que ofrece respecto a NutriMenu es la posibilidad de \textbf{variar los índices que se desean usar como referencia}, ya que en vez de utilizar unos valores estándar tiene en cuenta más elementos, como el peso, altura, edad... \cite{myfitnesspal:main}

Sin embargo, hasta el momento esta aplicación \textbf{no ofrece ninguna función orientada a los centros de restauración}, ya que está directamente enfocada a los usuarios que van a consumir los alimentos. Esto creo que es la gran ventaja de NutriMenu, ya que todas sus opciones de gestión te permiten ofrecerte los platos y alimentos del menú que vas a consumir en ese momento, en vez de tener que ser tú quien los añadas a la aplicación para poder obtener el informe nutricional.

\section{Yazio}

Yazio es otra aplicación móvil con un funcionamiento y modelo de negocio similar a MyFitnessPal, ya que también está únicamente enfocada al usuario consumidor de los productos, en vez de tener en cuenta también a los centros de restauración.

Este servicio también \textbf{dispone de su propia base de datos de alimentos}, sin embargo, en la mayoría de los casos la base de datos de MyFitnessPal resulta más nutrida y extensa.

Una ventaja de Yazio es que ofrece \textbf{planes de ayuno intermitente}, lo que puede hacer que esta aplicación sea más interesante para usuarios interesados en esta área.

Esta aplicación también \textbf{dispone de una oferta de recetas} para complementar la búsqueda de una nutrición equilibrada, lo que puede ser de gran ayuda a los usuarios que vayan a comer en casa \cite{yazio:main}. A pesar de que se aleja un poco de los objetivos de NutriMenu, que son permitir realizar informes nutricionales de los alimentos disponibles en los centros de restauración, esto mismo permite que ambas aplicaciones puedan llegar a ser complementarias.

\section{Generador de informes nutricionales de la restauración en centros universitarios}

Este Trabajo de Fin de Grado \cite{tfg-mariya:memoria}, realizado por Mariya Aleksandrova, es el punto de origen del que parte mi proyecto. 

Mariya desarrolló dos aplicaciones web que permiten crear informes nutricionales a partir de los menús disponibles en los centros de restauración de la Universidad de Burgos, así como también facilitar su gestión.

Este proyecto partió de otro TFG, el realizado por Joseba Hernando Moisén, en el que se desarrolló una aplicación de escritorio con un objetivo parecido: crear platos y menús a partir de los datos obtenidos de la base de datos de BEDCA, y crear informes nutricionales con estos, para poder ser usados en las cafeterías y restaurantes de la Universidad de Burgos.

El proyecto que se ha realizado en mi Trabajo de Fin de Grado es una tercera iteración sobre estos dos trabajos, y se ha buscado mantener los propósitos e ideas originales, pero intentando mejorarlos en la medida de lo posible para tener un producto lo más cercano a algo que se pueda poner en producción.