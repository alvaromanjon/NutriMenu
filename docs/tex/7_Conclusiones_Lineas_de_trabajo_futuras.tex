\capitulo{7}{Conclusiones y Líneas de trabajo futuras}

\section{Conclusiones}
Este proyecto me ha permitido ganar mucha experiencia, pues al final he terminado realizando un desarrollo completo en el que he tenido que pasar por todas las etapas, desde el plantear cómo estructurar los datos, a pensar cómo representarlos, para acabar en investigar de qué forma hacer que esta aplicación esté disponible de la forma más eficiente posible. Me ha gustado rotar por todos los roles, lo cual me ha proporcionado una visión más amplia sobre el funcionamiento general y la responsabilidad de cada parte.

Antes de este proyecto tenía poca experiencia en el ámbito del desarrollo web, especialmente en lo relacionado con el frontend, y esto me ha hecho crecer como desarrollador y tener una idea más clara de hacia dónde quiero orientar mi carrera profesional en el futuro.

Además, el desarrollar un trabajo estrechamente relacionado con la nutrición me ha hecho más consciente de su importancia para nuestra salud y de cómo las pequeñas decisiones diarias pueden tener un gran impacto a largo plazo.

\section{Líneas de trabajo futuras}
\begin{itemize}
\item Implementación de una fuente de datos de alérgenos, para incrementar la información disponible sobre cada alimento.
\item Implementación de algún método que permita mostrar los alimentos con valores nutricionales sin rellenar como nulos, no como 0, para evitar confusiones a la hora de ver los informes.
\item Securización de la API mediante el uso de tókenes, para impedir el acceso a ciertos \textit{endpoints} dependiendo del tipo de usuario que los consulte.
\item Mejorar la gestión de las sesiones en la aplicación web, para que se almacenen de forma segura y cifrada, y tengan un \textit{timeout} que al superarse haga que se cierre la sesión.
\item Implementación de distintos tests, como unitarios y de rendimiento tanto en el frontend como en el backend.
\item Añadir bebidas como nuevo tipo de platos a la hora de componer los menús.
\item Añadir paginación, filtrado y distintas opciones de ordenación en las tablas de gestión de los distintos elementos.
\item Añadir soporte de localización a la aplicación, para que pueda ser mostrada en otros idiomas.
\item Añadir funcionalidades de accesibilidad a la aplicación, para que sea más sencilla de usar por todos los usuarios
\end{itemize}


