\capitulo{1}{Introducción}

En la actualidad, cada vez se le da mayor importancia al hecho de llevar una alimentación saludable y equilibrada, puesto que es uno de los pilares clave para una vida longeva y libre de enfermedades. La Universidad de Burgos es consciente de esto, y es por ello por lo que forma parte de la Red Española de Universidades Promotoras de la Salud, y desde hace años se encuentra desarrollando el proyecto de Aula Campus Saludable, una sección que se encarga de promocionar hábitos saludables a todos los miembros de la Comunidad Universitaria.

Dentro de este área hay varias iniciativas y campañas que se han llevado a cabo, y uno de estos proyectos fue el del desarrollo de un conjunto de aplicaciones, cuyo objetivo era el de informar a los consumidores de los distintos centros de restauración de la Universidad de Burgos de los componentes nutricionales que contienen los platos que forman los menús disponibles. Este conjunto de aplicaciones también permitía la gestión de estos platos y menús, así como de los propios centros de restauración, sus empresas y sus usuarios.

La iniciativa para este nuevo proyecto es la de modernizar y mejorar este trabajo, para que su despliegue e implementación sean lo más sencillos posibles, así como intentar mejorar las aplicaciones webs lo máximo posible, manteniendo su propósito original.
