\capitulo{5}{Aspectos relevantes del desarrollo del proyecto}

\begin{comment}
Este apartado pretende recoger los aspectos más interesantes del desarrollo del proyecto, comentados por los autores del mismo.
Debe incluir desde la exposición del ciclo de vida utilizado, hasta los detalles de mayor relevancia de las fases de análisis, diseño e implementación.
Se busca que no sea una mera operación de copiar y pegar diagramas y extractos del código fuente, sino que realmente se justifiquen los caminos de solución que se han tomado, especialmente aquellos que no sean triviales.
Puede ser el lugar más adecuado para documentar los aspectos más interesantes del diseño y de la implementación, con un mayor hincapié en aspectos tales como el tipo de arquitectura elegido, los índices de las tablas de la base de datos, normalización y desnormalización, distribución en ficheros3, reglas de negocio dentro de las bases de datos (EDVHV GH GDWRV DFWLYDV), aspectos de desarrollo relacionados con el WWW...
Este apartado, debe convertirse en el resumen de la experiencia práctica del proyecto, y por sí mismo justifica que la memoria se convierta en un documento útil, fuente de referencia para los autores, los tutores y futuros alumnos.
\end{comment}

En este apartado se recogerán los aspectos de mayor relevancia que han surgido durante la realización de este proyecto, con el objetivo de comprender mejor el desarrollo y los desafíos encontrados a lo largo del camino.

\section{Inicio del proyecto}

Este proyecto, propuesto por mis tutores, surge como continuidad de otro Trabajo de Final de Grado, el realizado por Mariya Aleksandrova Stroyanova, en el que trabajó para crear una versión web y ampliar las funcionalidades de otro Trabajo de Final de Grado, el realizado por Joseba Fernando Moisén, que consistía en una aplicación de escritorio de la que se podían obtener los valores nutricionales de los alimentos de la base de datos de BEDCA. 

El Trabajo de Mariya, que es del que he partido, consta de dos aplicaciones web: una encargada de la generación y gestión de empresas, locales, usuarios, menús y platos, que sería la parte que controlarían las empresas encargadas de la gestión de las cafeterías de las distintas facultades de la Universidad de Burgos; y otra aplicación que sería a la que tendrían acceso los clientes, en la cual se puede realizar informes nutricionales a partir de los menús escogidos por el usuario.

Mi propuesta para este Trabajo de Final de Grado ha sido la de mejorar esta aplicación web, tanto a nivel de infraestructura y servicios, como a nivel de interactividad y usabilidad, haciéndola mucho más amigable a los dispositivos móviles, ya que es el lugar desde donde van a acceder la mayoría de usuarios que visiten la cafetería.


