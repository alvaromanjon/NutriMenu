\capitulo{2}{Objetivos del proyecto}

Los objetivos que se buscan con el desarrollo de este proyecto son los siguientes:

\section{Objetivos generales}
\begin{itemize}
	\item Implementar una infraestructura que sea agnóstica al hardware en el que se ejecute.
	\item Separar la infraestructura en servicios bien definidos y aislados.
	\item Desarrollar un sistema de orquestación del despliegue y hacer que este sea sencillo y escalable.
	\item Buscar una forma de poder acceder a nuevos datos nutricionales sin tener que tenerlos almacenados en local de forma obligatoria.
	\item Unificar las dos aplicaciones web en una, mejorar su interfaz y experiencia de usuario para que sean más sencillas de usar y hacer que sean funcionales en dispositivos móviles.
	\end{itemize}
	
\section{Objetivos técnicos}
\begin{itemize}
\item Contenerizar todos los servicios mediante Docker y Docker Compose.
\item Reestructurar la lógica del modelo de datos para separarla de la aplicación web, manteniendo el patrón Modelo - Vista - Controlador (MVC).
\item Desarrollar una API REST usando Spring Boot para que esta sea la vía por la que la aplicación interactúe con los datos.
\item Implementar un flujo de integración continua mediante el uso de GitHub Actions para comprobar que la infraestructura es funcional en todo momento.
\item Desarrollar una aplicación web con React que cumpla con todos los requisitos de las dos aplicaciones previas.
\item Hacer un diseño responsive con Bootstrap, que funcione en cualquier tamaño de pantalla.
\item Implementar la API externa de Nutritionix como fuente de composición nutricional de los alimentos.
\item Hacer uso de GitHub para la gestión del control de versiones de este proyecto, así como la administración de las tareas y los sprints.

\end{itemize}